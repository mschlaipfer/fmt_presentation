\documentclass{beamer}
\usepackage[utf8]{inputenc}
\usepackage{uniinput}
\usepackage{proof}
\usepackage[english]{babel}
\usepackage{xstring}
\usepackage{ulem}

\newcommand{\L}{\mathcal{L}}
\newcommand{\K}{\mathcal{K}}
\newcommand{\A}{\mathcal{A}}
\newcommand{\P}{\mathcal{P}}

\usetheme{Madrid}
\usecolortheme{seahorse}
\setbeamertemplate{navigation symbols}{}

%\title
%\author

\begin{document}

\begin{frame}
	\titlepage
\end{frame}

% \begin{frame}
% 	\frametitle{Inhalt}
% 	\tableofcontents
% \end{frame}

\begin{frame}
	\frametitle{Fagin's Theorem}
	Goal: Machine-independent characterization of complexity classes
	\begin{definition}
		Let $\mathcal{K}$ be a complexity class and $\mathcal{L}$ a logic. We say that $\L$ captures $\K$ iff the following holds:
		\begin{enumerate}
			\item For every sentence $Φ$ of $\L$ and every finite structure $\A$, deciding whether $\A \vDash Φ$ is in $\K$;
			\item For every property $\P$ of finite structures that can be decided in $\K$ there is a sentence $Φ$ of $\L$ such that $\A$ has property $\P$ iff $\A \vDash Φ$.
		\end{enumerate}
	\end{definition}
	\begin{theorem}[Fagin]
		$∃SO$ captures NP.
	\end{theorem}
\end{frame}


\end{document}